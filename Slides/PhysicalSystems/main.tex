\documentclass{beamer}

\input{settings.tex}


\title{Physical systems}
\subtitle{Math and modeling for high school}
\author{by Sergei Savin}
\centering
\date{Fall 2022}



\begin{document}
\maketitle


\begin{frame}{Free fall}
	% \framesubtitle{Part 1}
	\begin{flushleft}
		
		A unit mass is flying up with speed to $v$, such that $v(0) = 9.81$. Assuming that gravitational acceleration is $g = 9.81$, and initial position of the unit mass is $s(0) = 0$, when will it hit the ground?
		
		\bigskip
		
		Remembering Newton's law we get:
		
		$$ma = - mg$$
		
		We know \emph{acceleration is the derivative of velocity} and \emph{velocity is the derivative of position}:
		
		\begin{align}
			\dot v = a \\
			\dot s = v
		\end{align}
		
		
	\end{flushleft}
\end{frame}



\begin{frame}{Free fall}
	% \framesubtitle{Part 1}
	\begin{flushleft}
		
		For our problem we have:
		%
		\begin{align}
			\dot v = -g \\
			\dot s = v
		\end{align}
		
		From that, we can find that $v(t) = -g t + C_1$, and since $v(0) = 9.81$, we get:
		%
		\begin{align}
	v(0) = 0+ C_1 = 9.81
		\end{align}		
	
	Finally we get: $v(t) = - 9.81 t +  9.81$. From that we get:
	%
		\begin{align}
			\dot s = - 9.81 t+  9.81 \\
			s(t) = -\frac{1}{2} 9.81 t^2 +  9.81t + C_2 
		\end{align}			
	
	From $s(0) = 0$ we get $ C_2 = 0$, so:
	%
		\begin{align}
	s(t) = -\frac{1}{2} 9.81 t^2 +  9.81t 
		\end{align}					
		
	\end{flushleft}
\end{frame}



\begin{frame}{Free fall}
	% \framesubtitle{Part 1}
	\begin{flushleft}
		
		Given $s(t) = -\frac{1}{2} 9.81 t^2 +  9.81t $, we can try to find when the unit mass will hit the ground, i.e. when $s(t) = 0$:
		
		\begin{align}
	-\frac{1}{2} 9.81 t^2 +  9.81t = 0 \\
	t_1 = 0\\
	t_2 = 2
		\end{align}					
	
	And that is the answer!
	
	\bigskip
	
	Let us remember that this analysis will work just as well with numerical solution for the differential equation. Meaning, we are not limited to simple tasks with simple solution here.
		
	\end{flushleft}
\end{frame}



\begin{frame}{Sinking}
	% \framesubtitle{Part 1}
	\begin{flushleft}
		
		Consider a unit mass sinking in viscous liquid. The viscosity force is modeled as $f_v = \mu v = 4 v$. The buoyancy force is $f_b = 1$. The mass of the unit is $m = 0.5$, the gravitational acceleration $g = 10$. What is the maximum velocity that the unit mass can achieve?
		
		\bigskip
		
		We can model the system as
		
		\begin{align}
			m a = f_b - mg - \mu v
		\end{align}			
	
			As before, we know that $\dot v = a$, so $\dot v =  f_b - mg - \mu v$. When the unit mass achieves its maximal velocity, it means that $\dot v = 0$:
			
		\begin{align}
			f_b - mg - \mu v_{max} = 0 \\
			v_{max} = \frac{f_b - mg}{\mu} =  \frac{1 - 5}{4} = -1
		\end{align}				
		
	\end{flushleft}
\end{frame}


\begin{frame}{Sinking}
	% \framesubtitle{Part 1}
	\begin{flushleft}
		
		Notice that, given $\dot v =  f_b - mg - \mu v$ we can also answer questions such as
		
		\begin{itemize}
			\item When will $v > 0.9 v_{max}$?
			
			\item How far will the unit mass sink from point $t = 1$ to the point $t = 3$, given zero initial velocity?
		\end{itemize}
	
	\bigskip
				
				These questions can be easily answered if you solve the equation numerically.
		
	\end{flushleft}
\end{frame}




\begin{frame}{Flying}
	% \framesubtitle{Part 1}
	\begin{flushleft}
		
		Consider a body with mass $m = 2$, flying horizontally with propulsive force $f = 10$ from point $t = 0$ until point $t = 2$. After $t = 2$, the force ceases. Initial conditions $v(0) = 1$, $s(0) = 0$. When will the body pass 100 meter mark?
		
		\bigskip
		
		We can model the system as:
		
		\begin{align}
			\begin{cases}
				2 a = 10 \\
				2 a = 0
			\end{cases}
		\end{align}	
	
	We know that $\dot v = a$, so $2 \dot v = 10$, so:
	
		\begin{align}
				2 \dot v = 10 \\
				v = 5 t + C_1
		\end{align}		
	
		
	\end{flushleft}
\end{frame}




\begin{frame}{Flying}
	% \framesubtitle{Part 1}
	\begin{flushleft}
		
		
		Since $v(0) = 0 + C_1 = 1$, we get:
		
		\begin{align}
			v(t) = \dot s = 5 t + 1 \\
			s = 2.5 t^2 + t + C_2 
		\end{align}		
		
		Since $s(0) = 0$, so $C_2 = 0$:
		
		\begin{align}
			s = 2.5 t^2 + t 
		\end{align}			
		
		What is the position when $t = 2$:
		
		\begin{align}
	s(2) = 10 + 2 = 12 \\
	v(2) = 10 + 1 = 11
		\end{align}			
		
	\end{flushleft}
\end{frame}



\begin{frame}{Flying}
	% \framesubtitle{Part 1}
	\begin{flushleft}
		
		
		Now we can go back and consider what happens after $t = 2$:
		
		\begin{align}
	\dot v = 0 \\
	v = C_3\\
	v(2) = C_3 = 11
		\end{align}				
		
		And finally, the position:
		
				\begin{align}
			\dot s = 11 \\
			s = 11 t + C_4\\
			s(2) = 22 + C_4 = 12 \\
			C_4 = -10 \\
			s(t) = 11 t - 10
		\end{align}		
	
	So, when will $s(t) = 11 t - 10  = 100$? The answer is $t = 110 / 11 = 10$		
		
	\end{flushleft}
\end{frame}




\begin{frame}{Sliding down a slope}
	% \framesubtitle{Part 1}
	\begin{flushleft}
		
		Let us consider a body with mass $m = 1$, sliding down a slope with angle $\alpha = \pi / 4$ and friction coefficient $\mu = 0.5$ along first $\sqrt{2}$ meters, and 0.75 later. We use parameters: gravitational acceleration $g = 10$, $x(0) = 0$, $v(0) = 0$.
		
		\bigskip
		
		We can find acceleration as:
		
		\begin{align}
			m a = m g \sin (\alpha) - \mu N
		\end{align}				
	
	where $N$ is normal reaction force. We can find the reaction force as $N = mg \cos (\alpha)$. So we get:
	
	\begin{align}
		m a = m g \sin (\alpha) - \mu mg \cos (\alpha) \\
		a = g (\sin (\alpha) - \mu \cos (\alpha)) = 
		\begin{cases}
			\frac{10}{4} \sqrt{2} \ \  \text{if}  \ \ s \leq \sqrt{2} \\
			\frac{10}{8} \sqrt{2} \ \ \text{if}  \ \ s > \sqrt{2}
		\end{cases}
	\end{align}		
		
	\end{flushleft}
\end{frame}



\begin{frame}{Sliding down a slope}
	% \framesubtitle{Part 1}
	\begin{flushleft}
		
		Considering that derivative of the velocity is acceleration, we get:
		
		$$
			\dot v = 
			\begin{cases}
				\frac{10}{4} \sqrt{2} \ \  \text{if}  \ \ s \leq \sqrt{2} \\
				\frac{10}{8} \sqrt{2} \ \ \text{if}  \ \ s > \sqrt{2}
			\end{cases}
		$$
		
		The first segment is $\dot v = \frac{10}{4} \sqrt{2}$ and $v = \frac{10}{4} \sqrt{2} t + C_1$. We know that $v(0) = 0$, so $C_1 = 0$. Thus:
		
\begin{align}
\dot s = \frac{10}{4} \sqrt{2} t \\
s =  \frac{10}{8} \sqrt{2} t ^2 + C_2 \\
s(0) = 0 \\
C_2 = 0 \\
s(t) =  \frac{10}{8} \sqrt{2} t ^2 
\end{align}			
		
	\end{flushleft}
\end{frame}



\begin{frame}{Sliding down a slope}
	% \framesubtitle{Part 1}
	\begin{flushleft}
		
		Given $s(t) =  \frac{10}{8} \sqrt{2} t ^2 $, what is the time when $s = \sqrt{2}$?
		
\begin{align}
	 \sqrt{2} =  \frac{10}{8} \sqrt{2} t ^2 \\
	 t_1 = \sqrt{0.8} 
\end{align}			

With that, we can consider equation $\dot v = \frac{10}{8} \sqrt{2}$ with initial conditions $v(\sqrt{0.8}) =  \frac{10}{4} \sqrt{1.8}$ and $s(\sqrt{0.8}) = \sqrt{2}$.
		
	\end{flushleft}
\end{frame}


\begin{frame}{Thank you!}
\centerline{Lecture slides are available via Moodle.}
\bigskip
\centerline{You can help improve these slides at:}
\centerline{\mygit}
\bigskip
\centerline{Check Moodle for additional links, videos, textbook suggestions.}
\bigskip

\centerline{\textcolor{black}{\qrcode[height=1.6in]{https://github.com/SergeiSa/Extra-math-for-high-school}}}

\end{frame}

\end{document}
