\documentclass{beamer}

\input{settings.tex}


\title{Vector ODEs}
\subtitle{Math and modeling for high school}
\author{by Sergei Savin}
\centering
\date{Fall 2022}



\begin{document}
\maketitle


\begin{frame}{Scalar and vector equations}
	% \framesubtitle{Part 1}
	\begin{flushleft}
		
		Consider two differential equations:
		
		\begin{equation}
			\begin{cases}
				\dot x = -10x - 3t \\
				\dot y = -2y + t
			\end{cases}
		\end{equation}
	
		Introducing a vector variable $\bo{v} = [x, \ y]$ we can re-write the system as:
		
		\begin{equation}
		\dot{\bo{v}} = 
		\begin{bmatrix}
			-10 & 0 \\
			0 & -2
		\end{bmatrix}		
		\bo{v}
		+
		\begin{bmatrix}
			-3t \\
			t
		\end{bmatrix}
		\end{equation}		
		
		Notice that it is still two independent equations. But it doesn't have to be. Here is an example of a system of equations which are not interdependent:
		
		\begin{equation}
			\dot{\bo{v}} = 
			\begin{bmatrix}
				-10 & \textcolor{red}{2} \\
				0 &  -2
			\end{bmatrix}		
			\bo{v}
			+
			\begin{bmatrix}
				-3t \\
				t
			\end{bmatrix}
		\end{equation}		
		
	\end{flushleft}
\end{frame}




\begin{frame}{Vector ODEs}
	% \framesubtitle{Part 1}
	\begin{flushleft}
		
		In general, consider equation
		
		\begin{equation}
			\dot{\bo{v}} = 
			\bo{A}\bo{v}
			+
			\bo{c}
		\end{equation}		
		
		As before, we can discuss how the derivative of a variable in a ratio between the increase in its value and the time interval over which the increase has taken place:
		
		\begin{equation}
	\dot{\bo{v}} \approx
	\frac{\bo{v}(t + \Delta t) - \bo{v}(t)}{\Delta t}
		\end{equation}				
		
		So, we can make it into an \emph{integration scheme}:
		
		\begin{equation}
			\bo{v}(t + \Delta t) = \bo{v}(t) + (\bo{A}\bo{v}(t) + \bo{c}) \Delta t
		\end{equation}			
		
	\end{flushleft}
\end{frame}



\begin{frame}{Satellite problem}
	% \framesubtitle{Part 1}
	\begin{flushleft}
		
		When do system of ODEs become important?
		
		\bigskip
		
		Consider a problem: given a satellite orbiting Earth at the velocity $\bo{v}$, at the height $h$. Find its orbit.
		
		\bigskip
		
		Sounds quite intimidating, doesn't it?
		
	\end{flushleft}
\end{frame}




\begin{frame}{Satellite problem}
\framesubtitle{Newton's law}
	\begin{flushleft}
		
		First, let us remember the Newton's law:
		
		\begin{equation}
			m \dot{\bo{v}} = \bo{f}
		\end{equation}
		
		where $m$ is the mass of the satellite, $\bo{v}$ is its velocity and $\bo{f}$ is the sum of all external forces acting on the satellite.
		
		\bigskip
		
		Second, we assume that the only external force acting on the satellite is gravitational force.
		
	\end{flushleft}
\end{frame}



\begin{frame}{Satellite problem}
\framesubtitle{Gravity model}
	\begin{flushleft}
		
		For us it will be insufficient to assume that the gravity force is given by $mg$ directed towards the center of the Earth. Instead, we want to use a more precise model of gravitational force:
		
		\begin{equation}
			f = G \frac{m M}{h^2}
		\end{equation}
	
		where $f$ is the magnitude of the gravitational force, $m$ is the mass of the satellite, $M$ is the mass of the Earth and $h$ is the distance between the center of the Earth and the satellite.
	
		\bigskip
	
		Here are some of these constants:		
		\begin{itemize}
			\item $M = 5.972 \cdot 10^{24}$ kg
			\item $G = 6.673 \cdot 10^{-11}$ $\text{N m}^2/\text{kg}^2$
			\item radius of Earth is $6.371 \cdot 10^{6}$ m
		\end{itemize}
		
	\end{flushleft}
\end{frame}



\begin{frame}{Satellite problem}
 \framesubtitle{Position and gravity}
	\begin{flushleft}
		
		Let us assume that the position of the satellite relative to the center of the Earth, is given by the vector $\bo{r}$. How can we use this information to compute the magnitude of the gravity force?
		
		\bigskip
		
		Our strategy is to realize that $h = ||\bo{r}||$, and use it in the formula:
		
		\begin{equation}
			f = G \frac{m M}{||\bo{r}||^2}
		\end{equation}		 
		
	\end{flushleft}
\end{frame}



\begin{frame}{Satellite problem}
\framesubtitle{Direction of gravity}
	\begin{flushleft}
		
		But how do we give this force a direction? It always points towards Earth, so in the direction $-\bo{r}$. The gravity force should be pointing in the same direction. 
		\bigskip
		
		The way to ensure it, is to multiply the magnitude $f = G \frac{m M}{||\bo{r}||^2}$ by the direction unit vector $-\bo{r} /  ||\bo{r}||$:
		
		\begin{equation}
			f = -G \frac{m M}{||\bo{r}||^3} \bo{r}
		\end{equation}		 
		
	\end{flushleft}
\end{frame}


\begin{frame}{Satellite problem}
	\framesubtitle{Model}
	\begin{flushleft}
		
		Now we know the gravity force, and we can write the equations of motion for the satellite:
		
		\begin{equation}
			\begin{cases}
				\dot{\bo{r}} = \bo{v} \\
				m \dot{\bo{v}} = -G \frac{m M}{||\bo{r}||^3} \bo{r}
			\end{cases}
		\end{equation}
	
	Clearly we can divide the first equation by $m$, ridding us from this parameter - meaning the mass of the satellite does not play a role in this problem. This is true because we do not consider the effect of other forces than gravity, and we do not consider the effects the satellite has on the Earth.
		
	\end{flushleft}
\end{frame}



\begin{frame}[fragile]{Satellite problem}
	\framesubtitle{Code}
	\begin{flushleft}
		
		We can solve this system in MATLAB using the following code:
		
		\begin{lstlisting}[language=Matlab]
			G = 6.673 * 10^(-11);
			M = 5.972 * 10^24;
			r0 = randn(3, 1) * 7*10^6;
			v0 = randn(3, 1) * 1000;
			tspan = linspace(0, 20^4, 10^5);
			x0 = [r0; v0];
			odefun = @(t, x) [x(4:6); -G*M*x(1:3) / ... 
			((norm(x(1:3)))^3)];
			opts = odeset('Reltol',1e-13,'AbsTol',1e-14,'Stats','on');
			[t, x] = ode45(odefun,tspan,x0, opts);
			plot3(x(:, 1), x(:, 2), x(:, 3));
		\end{lstlisting}
		
		
		
	\end{flushleft}
\end{frame}



\begin{frame}{Satellite problem}
	\framesubtitle{Graph}
	\begin{flushleft}
		
		We plot the solution, and the result is shown below:
		
		% TODO: \usepackage{graphicx} required
		\begin{figure}
			\centering
			\includegraphics[width=0.9\linewidth]{Satellite}
			\caption{Orbit of a satellite, found by ode45; center of the Earth is given by a red dot}
			\label{fig:satellite}
		\end{figure}
		
		
	\end{flushleft}
\end{frame}




\begin{frame}{Thank you!}
\centerline{Lecture slides are available via Moodle.}
\bigskip
\centerline{You can help improve these slides at:}
\centerline{\mygit}
\bigskip
\centerline{Check Moodle for additional links, videos, textbook suggestions.}
\bigskip

\centerline{\textcolor{black}{\qrcode[height=1.6in]{https://github.com/SergeiSa/Extra-math-for-high-school}}}

\end{frame}

\end{document}
