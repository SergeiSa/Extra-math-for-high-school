\documentclass[12pt]{article}
\usepackage{blindtext}
\usepackage[a4paper, total={7.5in, 10in}]{geometry}

\usepackage[utf8]{inputenc}


\usepackage{amsmath,amssymb,amsfonts}
\usepackage{amsthm}  
\usepackage{mathrsfs}


\usepackage{xcolor}

% \usepackage{lscape}

\newcommand{\bo}[1] {\mathbf{#1}}
\newcommand{\R} {\mathbb{R}}
\newcommand{\norm}[1] {\left|\left|{#1}\right|\right|}
\newcommand{\T}{^\top}

% \newcommand{\col}[1] {\text{col}\left(#1\right)}
% \newcommand{\nulls}[1] {\text{null}\left(#1\right)}

\newcommand{\jac}[2] {\frac{\partial \bo{#1}}{\partial \bo{#2}}}

% \newcommand{\cos}[1] {\text{cos}\left(#1\right)}
% \newcommand{\sin}[1] {\text{cos}\left(#1\right)}

\newtheorem{assumption}{Assumption}
\newtheorem{theorem}{Theorem}


\definecolor{mypink}{RGB}{255, 30, 80}
\definecolor{mydarkred}{RGB}{160, 30, 30}
\definecolor{mylightred}{RGB}{255, 150, 150}
\definecolor{mylightgreyred}{RGB}{250, 240, 240}
\definecolor{myred}{RGB}{200, 110, 110}
\definecolor{myblackred}{RGB}{120, 40, 40}

\definecolor{myblue}{RGB}{240, 240, 255}
\definecolor{mydarkblue}{RGB}{60, 160, 255}
\definecolor{myblackblue}{RGB}{40, 40, 120}

\definecolor{mylime}{RGB}{200, 240, 240}
\definecolor{mygreen}{RGB}{0, 200, 0}
\definecolor{mylightgreen}{RGB}{205, 255, 200}
\definecolor{mylightgreygreen}{RGB}{240, 250, 240}
\definecolor{mydarkgreen}{RGB}{0, 200, 0}
\definecolor{myblackgreen}{RGB}{0, 100, 0}

\definecolor{mygray}{gray}{0.8}
% \definecolor{mydarkgray}{gray}{0.4}
\definecolor{mydarkgray}{RGB}{80, 80, 160}



%%%%%%%%%%%%%%%%%%%%%%%%%%%%%%%%%%%%%%%

\usepackage{tcolorbox}
\tcbuselibrary{skins,breakable}
\usetikzlibrary{shadings,shadows}

\newenvironment{myexampleblock}[1]{%
    \tcolorbox[beamer,%
    noparskip,breakable,
    colback=mylightgreygreen,colframe=myblackgreen,%
    colbacklower=mylime!75!mylightgreygreen,%
    title=#1]}%
    {\endtcolorbox}

\newenvironment{myalertblock}[1]{%
    \tcolorbox[beamer,%
    noparskip,breakable,
    colback=mylightgreyred,colframe=myblackred,%
    colbacklower=myred!75!mylightgreyred,%
    title=#1]}%
    {\endtcolorbox}

\newenvironment{myblock}[1]{%
    \tcolorbox[beamer,%
    noparskip,breakable,
    colback=myblue,colframe=myblackblue,%
    colbacklower=myblackblue!75!myblue,%
    title=#1]}%
    {\endtcolorbox}

\title{MyMathTemplate}
% \author{sergey89mtkgtu }
% \date{May 2022}

\begin{document}
% \maketitle

\section{Background}

There are 4 equations that govern this case:

1. Current position of the satellite is orthogonal to its orbit axis:

$$\bo{n}\T \bo{r} = 0$$

Second, its current velocity $\tau$ is orthogonal to both and:

$$\tau = \bo{n} \times \bo{r}$$

Third, the angle between the north pole axis and the orbit axis is $\varphi$:

$$\bo{n}\T \bo{z} = \cos(\varphi)$$

And finally, $||\bo{n}|| = 1$.

\section{Solution}

If $\bo{n} = [x_n, \ y_n, \ z_n]$, then $\bo{n}\T \bo{z} = \cos(\varphi)$ becomes:

$$z_n = \cos(\varphi)$$

If $\bo{r} = [x_r, \ y_r, \ z_r]$, then $\bo{n}\T \bo{r} = 0$ is equivalent to 

$$x_r x_n + y_r y_n = -z_r \cos(\varphi)$$

from which we get $x_n$:

$$x_n = - y_r/x_r y_n -z_r/x_r \cos(\varphi)$$

denoting $a = - y_r/x_r$ and $b = -z_r/x_r \cos(\varphi)$, we get:

$$ x_n = a y_n + b$$

Finally, the $||\bo{n}|| = 1$ gives us:

$$ (a y_n + b)^2 + y_n^2 + \cos^2(\varphi) = 1$$

$$ (a^2 + 1)y_n^2 +2ab y_n + b^2 = \sin^2(\varphi)$$



\end{document}
